\begin{abstract}
Ant foraging behavior is critical for understanding ecological dynamics and developing effective pest management strategies. However, quantifying this behavior is challenging due to the labor-intensive nature of manual counting, particularly in densely populated images. This study introduces an automated approach using computer vision to count ants and analyze their foraging patterns. Leveraging the YOLOv8 model, the system was calibrated and tested on datasets with varying imaging scenarios and densities. The results show that the system achieves an average precision and recall of up to 87.96\% and 87.78\%, respectively, with just 64 calibration images, provided both calibration and evaluation images share similar backgrounds. In cases where the evaluation images have more complex backgrounds than the calibration set, the system requires a larger calibration dataset to generalize effectively. Using 1,024 calibration images in such cases, the system achieved a precision of 83.60\% and a recall of 78.88\%. For particularly challenging scenarios where over a thousand ants appear in a single image, the system can still maintain satisfactory performance by slicing images into smaller patches, reaching a precision of 77.97\% and a recall of 71.36\%. Additionally, the system generates heatmaps that visualize the spatial distribution of ant activity over time. This spatial-temporal analysis deepens our understanding of ant behavior and pest control efforts. By automating the counting process and providing detailed behavioral insights, this study offers an efficient tool for researchers and pest control professionals to develop more effective strategies.
\end{abstract}

% \keywords{ Ant foraging \and computer vision \and behavioral ecology}
% Manually add footnotes for corresponding authors after the abstract

\footnotetext{\textsuperscript{*}Corresponding authors:  C. P. James Chen (\texttt{niche@vt.edu}) and Chin-Cheng Scotty Yang (\texttt{scottyyang@vt.edu})}