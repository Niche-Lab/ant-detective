\section{Conclusion}

This study successfully demonstrated the use of computer vision, specifically the YOLO model, to automate ant detection and counting, offering a more efficient and comprehensive alternative to manual methods. The findings reveal that the quantity of image data required for model calibration varies depending on the complexity of the background. In scenarios with consistent backgrounds, a relatively small calibration set proves sufficient, while more complex backgrounds necessitate a larger calibration set for effective model generalization.

The study also highlighted the effectiveness of image slicing in enhancing model performance when dealing with dense ant populations. By dividing the image into smaller patches, the model can overcome the limitations of standard feature map sizes and achieve higher accuracy in detecting individual ants.

The ability to track the precise location and size of each ant using the YOLO object detection format enables the generation of ant activity heatmaps, providing valuable insights into the spatial and temporal dynamics of ant foraging behavior. The integration of spatial and temporal information offers a deeper understanding of foraging patterns, potentially aiding in the development of more targeted and effective pest control strategies.
