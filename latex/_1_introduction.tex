\section{Introduction}

Ant foraging is probably one of the most studied themes in ant research because their behavioral repertoires are remarkably diverse \cite{Reeves2019Evolution}. Studying foraging behavior in ants can facilitate our understanding of ecological dynamics, as ants play vital roles in numerous ecosystems functioning including nutrient cycling, seed dispersal, etc. \cite{Parr2022Response}. Their foraging patterns can also reveal insights into social organization, communication, and adaptability to environmental changes including pathogen challenge \cite{Alciatore2021Immune} or pesticide exposure \cite{Thiel2016Sublethal}. Foraging behaviors are particularly crucial when developing an effective baiting strategy. This is because the success of baiting heavily relies upon the ants ingesting a lethal dose of the bait, which can be readily facilitated by taking advantage of better understanding of their foraging behavior \cite{Galante2024Acute}.

Quantifying long-term behaviors and investigating foraging preferences in ants typically involves manual counting across a large number of images. This process is generally labor-intensive, time-consuming and prone to human error, particularly when dealing with large numbers of ants. As a result, scaling up experiments to simulate more natural conditions, such as larger colony sizes and increased numbers of colonies, is challenging. For example, \cite{Hsu2018Viral} observed foraging behavior on a total of 18 red imported fire ant (\emph{Solenopsis invicta}) colonies, with each colony comprising approximately 3,000 individual ants. Each colony was offered four food resources with different macronutrient ratios. The number of foragers on each food source was manually estimated by analyzing photos taken by an automatic recording system that takes a picture on each food resource every 10 minutes for 32 hours, which resulted into a total of 13,824 photos that were processed by the study. Additionally, the number of ants counted in an image or at a specific time point may not fully capture the complexity of foraging behaviors, which often involve both spatial factors and the interaction between spatial and temporal dynamics, such as testing different sugar concentrations \cite{Sola2016Feeding} or bait types \cite{Du2023Foraging}. 
    
These challenges make manual counting a laborious task. While it can still yield results, tools that specifically tackle this challenge would facilitate counting process and thus the development of more effective pest control strategies. Indeed, relevant efforts have been made to automate the process of counting insects. Thresholding is a common method used to segment objects in images, enabling automated insect counting. It assumes that the object of interest exhibits distinct visual contrast with the background. For example, black fly adults were automatically counted by being placed on a Petri dish with a white background after collection from light traps \cite{Parker2020Using}. Similarly, \cite{Smythe2020Using} used a comparable method to count horn flies on cattle in the field, relying on the color contrast between the flies and the cattle’s skin.  
    
However, this method can be biased when the object and the background share similar visual properties. A more advanced approach involves using deep learning models that learn the morphological patterns of the object and are less sensitive to background appearance. For instance, the model YOLOv7 \cite{Wang2022YOLOv7} and YOLOv8 \cite{Jocher2023YOLO,Terven2023Comprehensive} were leveraged to count polyphagous moths in a customized trap \cite{Saradopoulos2023Image}. However, it can still be challenging to detect small objects like ants due to the limitations of current computer vision (CV) models, which are the primary approach for extracting information from images or videos. The challenges can be attributed to two main factors: pretraining data sources and model architecture. Typically, the pretraining data sources used to establish and train CV models are public datasets like COCO \cite{Lin2014Microsoft} and ImageNet \cite{Deng2009ImageNet}, designed for larger objects occupying a significant portion of an image. Since ants are small and often densely packed, standard models struggle to identify them accurately. The model architecture is another factor that limits the detection of small objects. Most CV models, such as YOLOv8, first resize input images to $640 \times 640$ pixels and then process them through several convolutional layers, resulting in progressively smaller feature maps. The smallest of these feature maps has a spatial dimension of $20 \times 20$ pixels. This downsampling means that an object smaller than $32 \times 32$ pixels (calculated by dividing 640 by 20) will be reduced to a single value in the smallest feature map. As a result, the model cannot effectively capture the fine spatial details necessary to distinguish small objects like ants, making accurate detection extremely difficult. 
    
To address the limitation, a study introduced Fully Convolutional Regression Networks (FCRNs) to count small insects, such as thrips and aphids, on leaves \cite{BereciartuaPerez2023Multiclass}. The FCRN architecture is a symmetric encoder-decoder network that overcomes the size limitation of feature maps by upsampling them to the original image size, preserving the spatial information necessary for accurate counting. Additionally, the output of the FCRN is a density map that shares the same spatial dimensions as the input image and contains only the objects’ centroids. Instead of predicting the exact size and location of each object, as in most standard object detection models, the FCRN estimates the probability of an object’s presence at each pixel location. The disadvantage of this approach is that it usually requires a large amount of training data to generalize well to new images due to the lack of pre-trained models on public datasets containing millions of images. Another strategy to improve the detection of small objects is slicing the input image into smaller patches and then feeding them separately into the model. This approach also overcomes the feature map size limitation, as it resizes the smaller patches to a larger size before processing \cite{Hong2021Automatic,BereciartuaPerez2023Multiclass}. 
    
Considering these challenges and the potential benefits of automated counting systems, this study explores the use of CV to automate the counting process and provide a more comprehensive analysis of both the spatial and temporal dimensions of ant foraging behavior. The study has three main objectives: (1) determining the quantity of image data required for the system to generalize to new images under similar or different imaging conditions, (2) examining the system’s performance in densely packed imaging scenarios, and (3) investigating how CV can enhance our understanding of the spatial and temporal aspects of ant foraging behaviors.
